\documentclass[12pt, oneside]{article}

\usepackage[left=3.5cm, top=2.5cm, bottom=2.5cm, right=2.5cm]{geometry}

\usepackage[utf8]{inputenc}
\usepackage[T1]{polski}
\usepackage[polish]{babel}

\usepackage[]{todonotes}

% temporarily for dark background
\usepackage{xcolor}
\definecolor{mybackground}{rgb}{0.15, 0.15, 0.13}
\definecolor{mytext}{rgb}{0.94, 0.94, 0.92}
% \pagecolor{mybackground}
% \color{mytext}

\begin{document}  
\thispagestyle{empty}
\begin{titlepage}
    \begin{center}

      \Large
	    \textbf{Uniwersytet Jagielloński w Krakowie}\vspace{0.2cm}\\ Wydział Fizyki, Astronomii i Informatyki Stosowanej
      \vspace*{1cm}
               
      \vspace{3cm}
      \Large
      \textbf{Wojciech Lepich}\\\vspace{0.5cm}
      \normalsize Nr albumu: 1146600\\
      \vspace{2cm}
      \Huge
      \textbf{Rozpoznawanie cyfr przez sieć neuronową zaimplementowaną na układzie FPGA}
      
      \vspace{1.5cm}
      \normalsize
      Praca licencjacka\\
      na kierunku Informatyka\\ \vspace{0.15cm}
        
      \vfill
      \vspace{2cm}
      \begin{minipage}{1\textwidth}
\begin{flushright}
Praca wykonana pod kierunkiem\\
dr. Grzegorza Korcyla\\
z Zakładu Technologii Informatycznych
\end{flushright}
\end{minipage}
        
        \vspace{2cm}
        \begin{center}
      Kraków 2020
        \end{center}
    \end{center}
\end{titlepage}

\newpage 
\thispagestyle{empty}
\vspace{2.5cm}
\begin{flushleft}
\large \textbf{Oświadczenie autora pracy}\vspace{0.6cm}\\
\end{flushleft}

\noindent Świadom odpowiedzialności prawnej oświadczam, że niniejsza praca dyplomowa została napisana przeze mnie samodzielnie i nie zawiera treści uzyskanych w sposób niezgodny z obowiązującymi przepisami.\\

\noindent Oświadczam również, że przedstawiona praca nie była wcześniej przedmiotem procedur związanych z uzyskaniem tytułu zawodowego w wyższej uczelni.
\vspace{2cm}
\begin{center}
\begin{tabular}{lr}
\ldots\ldots\ldots\ldots\ldots\ldots~~~~~~~~~~~~~~~~~~~~~~~~~~~~~~~~~~~~~~&
\ldots\ldots\ldots\ldots\ldots\ldots\ldots\ldots\ldots \\
{~~~~Kraków, dnia} & {Podpis autora pracy~~~~}
\end{tabular}
\end{center}
\vspace{5cm}
\begin{flushleft}
\large \textbf{Oświadczenie kierującego pracą}
\end{flushleft}

\noindent Potwierdzam, że niniejsza praca została przygotowana pod moim kierunkiem i~kwalifikuje się do przedstawienia jej w postępowaniu o nadanie tytułu zawodowego.
\vspace{2cm}
\begin{center}
\begin{tabular}{lr}
\ldots\ldots\ldots\ldots\ldots\ldots~~~~~~~~~~~~~~~~~~~~~~~~~~~~~~~~~~~~~~&
\ldots\ldots\ldots\ldots\ldots\ldots\ldots\ldots\ldots \\
{~~~~Kraków, dnia} & {Podpis kierującego pracą~~}
\end{tabular}
\end{center}
\vfill

\newpage
\tableofcontents

\newpage
\section{Wstęp}
Tutaj wstęp

\newpage
\section{Teoria}
\subsection{Architektura FPGA}
Field-programmable gate array (FPGA) to układy scalone, które mogą być
elektronicznie przeprogramowane bez potrzeby demontażu samego układu
z urządzenia. W porównaniu do układów ASIC znacznie taniej zaprojektować
pierwszy działający układ. Elastyczna natura układów FPGA wiąże się z większym
zużyciem powierzchni krzemu, opóźnień oraz zużycia energii.
\todo{Dodać przypis} (FPGA architecture: survey and challenges)

Podstawowa struktura układów FPGA składa się z różnych bloków logicznych,
które mogą być łączone zależnie od projektu. Przykładami takich bloków są:
DSP (jednostka przeprowadzająca obliczenia dodawania/mnożenia),
LUT (look-up table, de facto tablica prawdy dowolnej funkcji boolowskiej),
Flip Flop (przechowują wynik LUT), BRAM (block RAM, pamięć dwuportowa,
jest w stanie przechowywać względnie dużą ilość danych).

Układy FPGA przeważnie pracują na kilku-, kilkunastukrotnie niższych
częstotliwościach niż CPU. \todo{Przypis} Wysoką wydajność zawdzięczają
zrównolegleniu obliczeń.

\subsection{Przetwarzanie obrazu}
Cyfrowe przetwarzanie obrazu jest problemem wymagającym dużych mocy
obliczeniowych ze względu na ilość danych do przetworzenia. Nieskompresowany
kolorowy obraz z pikselami w formacie RGB (po 8 bitów na kolor) o wysokości
720 pikseli i szerokości 1280 pikseli to 22118400 bitów (\(\approx \) 2,5MB). Obraz
przetwarzany w czasie rzeczywistym, na przykład z kamery, zwielokrotnia tę
liczbę o liczbę klatek na sekundę (przy trzydziestu klatkach na sekundę liczba
danych rośnie do około 79 megabajtów na sekundę). Należy również pamiętać, że
dane są dwuwymiarowe co jest ważne przy problemach związanych z rozpoznawaniem
wzorców, klasyfikacją przedmiotów na obrazie, filtrowania w celu rozmazania lub
wyostrzenia obrazów, itp.

\subsubsection{Formaty pikseli}
Jest wiele modeli przestrzeni barw (a co za tym idzie, sposobów kodowania
pikseli) między innymi:
\begin{itemize}
  \item RGB, używany w aparatach, skanerach, telewizorach
  \item CMYK, używany w druku wielobarwnym
  \item HSV
  \item YUV
\end{itemize}
Składowe dwóch ostatnich przestrzeni barw oddzielają informację o jasności
od informacji o kolorach. Model barw YUV składa się z kanału luminacji Y
oraz kanałów kodujących barwę U oraz V, są to kolejno składowa niebieska
i składowa czerwona. W projekcie użyty jest format pikseli YUY2 (znany też
pod nazwą YUYV), w którym na dwa piksele przypadają 32 bity.
\todo{Będzie obrazek ze schematem}
Licząc od najstarszego bitu pierwsze osiem bitów przypada na Y0, to jest
luminacja pierwszego piksela, następne osiem bitów na U0, kolejne osiem bitów
to luminacja drugiego piksela, a pozostałe bity to składowa czerwona V0.
Dla obydwóch pikseli składowe U i V są wspólne. Co istotne w projekcie,
łatwo oddzielić luminację, która jest używana w przetwarzaniu obrazu.

\subsection{Sieci neuronowe}
Sztuczna sieć neuronowa (SSN) jest modelem zdolnym do odwzorowania złożonych
funkcji. Najprostsze sieci są zbudowane ze sztucznych neuronów, z których każdy
posiada wiele wejść oraz jedno wyjście, które może być połączone z wejściami
wielu innych neuronów. Każde z wejść neuronu jest związane ze znalezioną
w procesie trenowania wagą. Wartość wyjścia to obliczony wynik funkcji aktywacji
z sumy ważonych wejść. Sieć może mieć wiele warstw neuronów ukrytych, których
wejściami są wyjścia neuronów z poprzedniej warstwy. 

Sieci neuronowe są stosowane \todo{Przypis} w problemach
związanych z predykcją, klasyfikacją, przetwarzaniem i analizowaniem
danych. Do ich zastosowania nie jest potrzebna znajomość algorytmu rozwiązania
danego problemu. Obliczenia w sieciach są wykonywane równolegle w każdej
warstwie, dzięki czemu implementacja sieci na układzie FPGA może działać
wielokrotnie szybciej niż na CPU, pomimo niższej częstotliwości układu.

\newpage
\section{Opis projektu}

\subsection{Zarys projektu}
Celem projektu jest napisanie wtyczki do frameworka GStreamer
wykorzystującej sieć neuronową do rozpoznawania cyfr w czasie rzeczywistym 
na układzie Xilinx Zynq MPSoC. Wtyczka jest następnie wykorzystana
w zdefiniowanym potoku uruchomionym przy pomocy narzędzia gst-launch-1.0.
Zadaniami spoczywającymi na innych elementach potoku jest obsługa kamery,
kadrowanie i skalowanie obrazu oraz wyświetlenie go na końcowym urządzeniu.

\subsection{Platforma}
\todo{Zdjęcia stanowiska}
Sprzęt wykorzystany w projekcie to Xilinx Zynq UltraScale+ MPSoC ZCU104.
Na jednym układzie znajduje się czterordzeniowy procesor ARM Cortex-A53,
dwurdzeniowy procesor ARM Cortex-R5, układ graficzny Mali-400 oraz zasoby
FPGA. Całość projektu została oparta o platformę Xilinx reVISION. Przetwarzane
dane dostarczane są z kamery USB, która była dołączona w zestawie z płytą Zynq.
Urządzeniem końcowym jest telewizor połączony przewodem HDMI z płytą.

\subsection{Sieć neuronowa}
Jak jest zbudowana, jak uczona, dlaczego taka a nie inna; problemy z LeNet-5

\subsection{hls4ml}
Co to za framework i dlaczego taki fajny, dostosowanie precyzji z 
hls4ml.profiling. Tutaj też o dostosowaniu sieci, tzn.~progowanie „białych” 
pikseli itd.

\subsection{Gstreamer}
Do czego służy, jaki zbudowałem pipeline

\subsection{Używanie sieci}
Czyli synteza sieci i zrobienie z niej biblioteki statycznej „.a” 

\subsection{Część neuralnet}
Czytanie obrazu, podział na część luma i chroma, wywołanie funkcji sieci,
zapis z powrotem, synteza do biblioteki dzielonej „.so”

\subsection{Część gstsdxnet}
De facto plugin gstreamera, w którym są wywoływane funkcje z biblioteki
dzielonej neuralnet.so, 

\subsection{Małe podsumowanie}


\newpage
\section{Wyniki i dyskusja}

\subsection{Ewaulacja modelu}
Wyniki z samego pythona z danymi testowymi z mnista

\subsection{Symulacja}
Tutaj wyniki z symulacji z danymi testowymi z mnista

\subsection{Dane rzeczywiste}
Wyniki z kamerki. Zdjęcia danych testowych, co wpływa na wynik, czy wszystko
rozpoznaje itd,

\newpage
\section{Podsumowanie}
W projekcie zostało zrobione to i to. Wyszło to tak i tak. Problem sprawiło
tamto i owamto. Można to poprawić w ten sposób. Można część funkcjonalności
z pipeline przenieść na fpga (w końcu przetwarzanie obrazu na fpga jest szybkie)

\end{document}