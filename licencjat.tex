\documentclass[12pt, oneside]{article}

\usepackage[left=3.5cm, top=2.5cm, bottom=2.5cm, right=2.5cm]{geometry}

\usepackage[utf8]{inputenc}
\usepackage[T1]{polski}
\usepackage[polish]{babel}

\usepackage[]{todonotes}

% temporarily for dark background
\usepackage{xcolor}
\definecolor{mybackground}{rgb}{0.15, 0.15, 0.13}
\definecolor{mytext}{rgb}{0.94, 0.94, 0.92}
% \pagecolor{mybackground}
% \color{mytext}

\begin{document}  
\thispagestyle{empty}
\begin{titlepage}
    \begin{center}

      \Large
	    \textbf{Uniwersytet Jagielloński w Krakowie}\vspace{0.2cm}\\ Wydział Fizyki, Astronomii i Informatyki Stosowanej
      \vspace*{1cm}
               
      \vspace{3cm}
      \Large
      \textbf{Wojciech Lepich}\\\vspace{0.5cm}
      \normalsize Nr albumu: 1146600\\
      \vspace{2cm}
      \Huge
      \textbf{Rozpoznawanie cyfr przez sieć neuronową zaimplementowaną na układzie FPGA}
      
      \vspace{1.5cm}
      \normalsize
      Praca licencjacka\\
      na kierunku Informatyka\\ \vspace{0.15cm}
        
      \vfill
      \vspace{2cm}
      \begin{minipage}{1\textwidth}
\begin{flushright}
Praca wykonana pod kierunkiem\\
dr. Grzegorza Korcyla\\
z Zakładu Technologii Informatycznych
\end{flushright}
\end{minipage}
        
        \vspace{2cm}
        \begin{center}
      Kraków 2020
        \end{center}
    \end{center}
\end{titlepage}

\newpage 
\thispagestyle{empty}
\vspace{2.5cm}
\begin{flushleft}
\large \textbf{Oświadczenie autora pracy}\vspace{0.6cm}\\
\end{flushleft}

\noindent Świadom odpowiedzialności prawnej oświadczam, że niniejsza praca dyplomowa została napisana przeze mnie samodzielnie i nie zawiera treści uzyskanych w sposób niezgodny z obowiązującymi przepisami.\\

\noindent Oświadczam również, że przedstawiona praca nie była wcześniej przedmiotem procedur związanych z uzyskaniem tytułu zawodowego w wyższej uczelni.
\vspace{2cm}
\begin{center}
\begin{tabular}{lr}
\ldots\ldots\ldots\ldots\ldots\ldots~~~~~~~~~~~~~~~~~~~~~~~~~~~~~~~~~~~~~~&
\ldots\ldots\ldots\ldots\ldots\ldots\ldots\ldots\ldots \\
{~~~~Kraków, dnia} & {Podpis autora pracy~~~~}
\end{tabular}
\end{center}
\vspace{5cm}
\begin{flushleft}
\large \textbf{Oświadczenie kierującego pracą}
\end{flushleft}

\noindent Potwierdzam, że niniejsza praca została przygotowana pod moim kierunkiem i~kwalifikuje się do przedstawienia jej w postępowaniu o nadanie tytułu zawodowego.
\vspace{2cm}
\begin{center}
\begin{tabular}{lr}
\ldots\ldots\ldots\ldots\ldots\ldots~~~~~~~~~~~~~~~~~~~~~~~~~~~~~~~~~~~~~~&
\ldots\ldots\ldots\ldots\ldots\ldots\ldots\ldots\ldots \\
{~~~~Kraków, dnia} & {Podpis kierującego pracą~~}
\end{tabular}
\end{center}
\vfill

\newpage
\tableofcontents

\newpage
\section{Wstęp}
Tutaj wstęp

\newpage
\section{Teoria}
\subsection{Architektura FPGA}
Field-programmable gate array (FPGA) to układy scalone, które mogą być
elektronicznie przeprogramowane bez potrzeby demontażu samego układu
z urządzenia. W porównaniu do układów ASIC znacznie taniej zaprojektować
pierwszy działający układ. Elastyczna natura układów FPGA wiąże się z większym
zużyciem powierzchni krzemu, opóźnień oraz zużycia energii.
\todo{Dodać przypis} (FPGA architecture: survey and challenges)

Podstawowa struktura układów FPGA składa się z różnych bloków logicznych,
które mogą być łączone zależnie od projektu. Przykładami takich bloków są:
DSP (jednostka przeprowadzająca obliczenia dodawania/mnożenia),
LUT (look-up table, de facto tablica prawdy dowolnej funkcji boolowskiej),
Flip Flop (przechowują wynik LUT), BRAM (block RAM, pamięć dwuportowa,
jest w stanie przechowywać względnie dużą ilość danych).

Układy FPGA przeważnie pracują na kilku-, kilkunastukrotnie niższych
częstotliwościach niż CPU. \todo{Przypis}Wysoką wydajność zawdzięczają
zrównolegleniu obliczeń.

\subsection{Przetwarzanie obrazu}
Formaty pikseli, jak są zbudowane, lorem ipsum.

\subsection{Sieci neuronowe}
Z czym się to je, jak działają, jak wygląda trenowanie

\newpage
\section{Opis projektu}

\subsection{Zarys projektu}
Jakich narzędzi korzystałem, co chcę osiągnąć, elo 420.

\subsection{Platforma}
Zcu104 cośtam. Oparty o reVISION \texttrademark{}. Zdjęcia

\subsection{Sieć neuronowa}
Jak jest zbudowana, jak uczona, dlaczego taka a nie inna; problemy z LeNet-5

\subsection{hls4ml}
A co to za framework i dlaczego taki fajny, dostosowanie precyzji z profiling
Tutaj też dopisz o dostosowaniu sieci, tzn.\ o progu

\subsection{Gstreamer}
Do czego służy, jaki zbudowałem pipeline

\subsection{Używanie sieci}
Czyli synteza sieci i zrobienie z niej biblioteki statycznej „.a” 

\subsection{Część neuralnet}
Czytanie obrazu, podział na część luma i chroma, wywołanie funkcji sieci,
zapis z powrotem, synteza do biblioteki dzielonej „.so”

\subsection{Część gstsdxnet}
De facto plugin gstreamera, w którym są wywoływane funkcje z biblioteki
dzielonej neuralnet.so, 

\subsection{Małe podsumowanie}


\newpage
\section{Wyniki i dyskusja}

\subsection{Ewaulacja modelu}
Wyniki z samego pythona z danymi testowymi z mnista

\subsection{Symulacja}
Tutaj wyniki z symulacji z danymi testowymi z mnista

\subsection{Dane rzeczywiste}
No i tutaj wyniki z kamerki. Myślę, że nie trzeba robić jakichś spisów wielkich,
wystarczy opisać co dobrze odczytało, co źle, dlaczego, jakie sieć ma problemy,
przygotowanie danych (zdjęcia!), jak poprawić skuteczność działania (LeNet xD)

\newpage
\section{Podsumowanie}
W projekcie zostało zrobione to i to. Wyszło to tak i tak. Problem sprawiło
tamto i owamto. Można to poprawić w ten sposób. Można część funkcjonalności
z pipeline przenieść na fpga (sam pisałem, że przetwarzanie obrazu na fpga
jest gicior).

\end{document}